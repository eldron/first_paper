
%% bare_conf.tex
%% V1.4b
%% 2015/08/26
%% by Michael Shell
%% See:
%% http://www.michaelshell.org/
%% for current contact information.
%%
%% This is a skeleton file demonstrating the use of IEEEtran.cls
%% (requires IEEEtran.cls version 1.8b or later) with an IEEE
%% conference paper.
%%
%% Support sites:
%% http://www.michaelshell.org/tex/ieeetran/
%% http://www.ctan.org/pkg/ieeetran
%% and
%% http://www.ieee.org/

%%*************************************************************************
%% Legal Notice:
%% This code is offered as-is without any warranty either expressed or
%% implied; without even the implied warranty of MERCHANTABILITY or
%% FITNESS FOR A PARTICULAR PURPOSE!
%% User assumes all risk.
%% In no event shall the IEEE or any contributor to this code be liable for
%% any damages or losses, including, but not limited to, incidental,
%% consequential, or any other damages, resulting from the use or misuse
%% of any information contained here.
%%
%% All comments are the opinions of their respective authors and are not
%% necessarily endorsed by the IEEE.
%%
%% This work is distributed under the LaTeX Project Public License (LPPL)
%% ( http://www.latex-project.org/ ) version 1.3, and may be freely used,
%% distributed and modified. A copy of the LPPL, version 1.3, is included
%% in the base LaTeX documentation of all distributions of LaTeX released
%% 2003/12/01 or later.
%% Retain all contribution notices and credits.
%% ** Modified files should be clearly indicated as such, including  **
%% ** renaming them and changing author support contact information. **
%%*************************************************************************


% *** Authors should verify (and, if needed, correct) their LaTeX system  ***
% *** with the testflow diagnostic prior to trusting their LaTeX platform ***
% *** with production work. The IEEE's font choices and paper sizes can   ***
% *** trigger bugs that do not appear when using other class files.       ***                          ***
% The testflow support page is at:
% http://www.michaelshell.org/tex/testflow/



\documentclass[conference]{IEEEtran}
% Some Computer Society conferences also require the compsoc mode option,
% but others use the standard conference format.
%
% If IEEEtran.cls has not been installed into the LaTeX system files,
% manually specify the path to it like:
% \documentclass[conference]{../sty/IEEEtran}





% Some very useful LaTeX packages include:
% (uncomment the ones you want to load)


% *** MISC UTILITY PACKAGES ***
%
%\usepackage{ifpdf}
% Heiko Oberdiek's ifpdf.sty is very useful if you need conditional
% compilation based on whether the output is pdf or dvi.
% usage:
% \ifpdf
%   % pdf code
% \else
%   % dvi code
% \fi
% The latest version of ifpdf.sty can be obtained from:
% http://www.ctan.org/pkg/ifpdf
% Also, note that IEEEtran.cls V1.7 and later provides a builtin
% \ifCLASSINFOpdf conditional that works the same way.
% When switching from latex to pdflatex and vice-versa, the compiler may
% have to be run twice to clear warning/error messages.






% *** CITATION PACKAGES ***
%
%\usepackage{cite}
% cite.sty was written by Donald Arseneau
% V1.6 and later of IEEEtran pre-defines the format of the cite.sty package
% \cite{} output to follow that of the IEEE. Loading the cite package will
% result in citation numbers being automatically sorted and properly
% "compressed/ranged". e.g., [1], [9], [2], [7], [5], [6] without using
% cite.sty will become [1], [2], [5]--[7], [9] using cite.sty. cite.sty's
% \cite will automatically add leading space, if needed. Use cite.sty's
% noadjust option (cite.sty V3.8 and later) if you want to turn this off
% such as if a citation ever needs to be enclosed in parenthesis.
% cite.sty is already installed on most LaTeX systems. Be sure and use
% version 5.0 (2009-03-20) and later if using hyperref.sty.
% The latest version can be obtained at:
% http://www.ctan.org/pkg/cite
% The documentation is contained in the cite.sty file itself.






% *** GRAPHICS RELATED PACKAGES ***
%
\ifCLASSINFOpdf
  % \usepackage[pdftex]{graphicx}
  % declare the path(s) where your graphic files are
  % \graphicspath{{../pdf/}{../jpeg/}}
  % and their extensions so you won't have to specify these with
  % every instance of \includegraphics
  % \DeclareGraphicsExtensions{.pdf,.jpeg,.png}
\else
  % or other class option (dvipsone, dvipdf, if not using dvips). graphicx
  % will default to the driver specified in the system graphics.cfg if no
  % driver is specified.
  % \usepackage[dvips]{graphicx}
  % declare the path(s) where your graphic files are
  % \graphicspath{{../eps/}}
  % and their extensions so you won't have to specify these with
  % every instance of \includegraphics
  % \DeclareGraphicsExtensions{.eps}
\fi
% graphicx was written by David Carlisle and Sebastian Rahtz. It is
% required if you want graphics, photos, etc. graphicx.sty is already
% installed on most LaTeX systems. The latest version and documentation
% can be obtained at:
% http://www.ctan.org/pkg/graphicx
% Another good source of documentation is "Using Imported Graphics in
% LaTeX2e" by Keith Reckdahl which can be found at:
% http://www.ctan.org/pkg/epslatex
%
% latex, and pdflatex in dvi mode, support graphics in encapsulated
% postscript (.eps) format. pdflatex in pdf mode supports graphics
% in .pdf, .jpeg, .png and .mps (metapost) formats. Users should ensure
% that all non-photo figures use a vector format (.eps, .pdf, .mps) and
% not a bitmapped formats (.jpeg, .png). The IEEE frowns on bitmapped formats
% which can result in "jaggedy"/blurry rendering of lines and letters as
% well as large increases in file sizes.
%
% You can find documentation about the pdfTeX application at:
% http://www.tug.org/applications/pdftex





% *** MATH PACKAGES ***
%
%\usepackage{amsmath}
% A popular package from the American Mathematical Society that provides
% many useful and powerful commands for dealing with mathematics.
%
% Note that the amsmath package sets \interdisplaylinepenalty to 10000
% thus preventing page breaks from occurring within multiline equations. Use:
%\interdisplaylinepenalty=2500
% after loading amsmath to restore such page breaks as IEEEtran.cls normally
% does. amsmath.sty is already installed on most LaTeX systems. The latest
% version and documentation can be obtained at:
% http://www.ctan.org/pkg/amsmath





% *** SPECIALIZED LIST PACKAGES ***
%
\usepackage{algorithmic}
% algorithmic.sty was written by Peter Williams and Rogerio Brito.
% This package provides an algorithmic environment fo describing algorithms.
% You can use the algorithmic environment in-text or within a figure
% environment to provide for a floating algorithm. Do NOT use the algorithm
% floating environment provided by algorithm.sty (by the same authors) or
% algorithm2e.sty (by Christophe Fiorio) as the IEEE does not use dedicated
% algorithm float types and packages that provide these will not provide
% correct IEEE style captions. The latest version and documentation of
% algorithmic.sty can be obtained at:
% http://www.ctan.org/pkg/algorithms
% Also of interest may be the (relatively newer and more customizable)
% algorithmicx.sty package by Szasz Janos:
% http://www.ctan.org/pkg/algorithmicx




% *** ALIGNMENT PACKAGES ***
%
%\usepackage{array}
% Frank Mittelbach's and David Carlisle's array.sty patches and improves
% the standard LaTeX2e array and tabular environments to provide better
% appearance and additional user controls. As the default LaTeX2e table
% generation code is lacking to the point of almost being broken with
% respect to the quality of the end results, all users are strongly
% advised to use an enhanced (at the very least that provided by array.sty)
% set of table tools. array.sty is already installed on most systems. The
% latest version and documentation can be obtained at:
% http://www.ctan.org/pkg/array


% IEEEtran contains the IEEEeqnarray family of commands that can be used to
% generate multiline equations as well as matrices, tables, etc., of high
% quality.




% *** SUBFIGURE PACKAGES ***
%\ifCLASSOPTIONcompsoc
%  \usepackage[caption=false,font=normalsize,labelfont=sf,textfont=sf]{subfig}
%\else
%  \usepackage[caption=false,font=footnotesize]{subfig}
%\fi
% subfig.sty, written by Steven Douglas Cochran, is the modern replacement
% for subfigure.sty, the latter of which is no longer maintained and is
% incompatible with some LaTeX packages including fixltx2e. However,
% subfig.sty requires and automatically loads Axel Sommerfeldt's caption.sty
% which will override IEEEtran.cls' handling of captions and this will result
% in non-IEEE style figure/table captions. To prevent this problem, be sure
% and invoke subfig.sty's "caption=false" package option (available since
% subfig.sty version 1.3, 2005/06/28) as this is will preserve IEEEtran.cls
% handling of captions.
% Note that the Computer Society format requires a larger sans serif font
% than the serif footnote size font used in traditional IEEE formatting
% and thus the need to invoke different subfig.sty package options depending
% on whether compsoc mode has been enabled.
%
% The latest version and documentation of subfig.sty can be obtained at:
% http://www.ctan.org/pkg/subfig




% *** FLOAT PACKAGES ***
%
%\usepackage{fixltx2e}
% fixltx2e, the successor to the earlier fix2col.sty, was written by
% Frank Mittelbach and David Carlisle. This package corrects a few problems
% in the LaTeX2e kernel, the most notable of which is that in current
% LaTeX2e releases, the ordering of single and double column floats is not
% guaranteed to be preserved. Thus, an unpatched LaTeX2e can allow a
% single column figure to be placed prior to an earlier double column
% figure.
% Be aware that LaTeX2e kernels dated 2015 and later have fixltx2e.sty's
% corrections already built into the system in which case a warning will
% be issued if an attempt is made to load fixltx2e.sty as it is no longer
% needed.
% The latest version and documentation can be found at:
% http://www.ctan.org/pkg/fixltx2e


%\usepackage{stfloats}
% stfloats.sty was written by Sigitas Tolusis. This package gives LaTeX2e
% the ability to do double column floats at the bottom of the page as well
% as the top. (e.g., "\begin{figure*}[!b]" is not normally possible in
% LaTeX2e). It also provides a command:
%\fnbelowfloat
% to enable the placement of footnotes below bottom floats (the standard
% LaTeX2e kernel puts them above bottom floats). This is an invasive package
% which rewrites many portions of the LaTeX2e float routines. It may not work
% with other packages that modify the LaTeX2e float routines. The latest
% version and documentation can be obtained at:
% http://www.ctan.org/pkg/stfloats
% Do not use the stfloats baselinefloat ability as the IEEE does not allow
% \baselineskip to stretch. Authors submitting work to the IEEE should note
% that the IEEE rarely uses double column equations and that authors should try
% to avoid such use. Do not be tempted to use the cuted.sty or midfloat.sty
% packages (also by Sigitas Tolusis) as the IEEE does not format its papers in
% such ways.
% Do not attempt to use stfloats with fixltx2e as they are incompatible.
% Instead, use Morten Hogholm'a dblfloatfix which combines the features
% of both fixltx2e and stfloats:
%
% \usepackage{dblfloatfix}
% The latest version can be found at:
% http://www.ctan.org/pkg/dblfloatfix




% *** PDF, URL AND HYPERLINK PACKAGES ***
%
%\usepackage{url}
% url.sty was written by Donald Arseneau. It provides better support for
% handling and breaking URLs. url.sty is already installed on most LaTeX
% systems. The latest version and documentation can be obtained at:
% http://www.ctan.org/pkg/url
% Basically, \url{my_url_here}.




% *** Do not adjust lengths that control margins, column widths, etc. ***
% *** Do not use packages that alter fonts (such as pslatex).         ***
% There should be no need to do such things with IEEEtran.cls V1.6 and later.
% (Unless specifically asked to do so by the journal or conference you plan
% to submit to, of course. )


% correct bad hyphenation here
\hyphenation{op-tical net-works semi-conduc-tor}


\begin{document}
%
% paper title
% Titles are generally capitalized except for words such as a, an, and, as,
% at, but, by, for, in, nor, of, on, or, the, to and up, which are usually
% not capitalized unless they are the first or last word of the title.
% Linebreaks \\ can be used within to get better formatting as desired.
% Do not put math or special symbols in the title.
%\title{Bare Demo of IEEEtran.cls\\ for IEEE Conferences}
\title{Implementation of TCP Large Receive Offload on Multi-core NPU Platform}

% author names and affiliations
% use a multiple column layout for up to three different
% affiliations
\author{\IEEEauthorblockN{Li Jie}
\IEEEauthorblockA{School of Computer\\
National University of\\Defense Technology\\
Changsha, China\\
Email: eldron@163.com}
\and
\IEEEauthorblockN{Chen Shuhui}
\IEEEauthorblockA{School of Computer\\
National University of\\Defense Technology\\
Email: homer@thesimpsons.com}
\and
\IEEEauthorblockN{Su Jinshu}
\IEEEauthorblockA{School of Computer\\
National University of\\Defense Technology\\
Telephone: (800) 555--1212\\
Fax: (888) 555--1212}}

% conference papers do not typically use \thanks and this command
% is locked out in conference mode. If really needed, such as for
% the acknowledgment of grants, issue a \IEEEoverridecommandlockouts
% after \documentclass

% for over three affiliations, or if they all won't fit within the width
% of the page, use this alternative format:
%
%\author{\IEEEauthorblockN{Michael Shell\IEEEauthorrefmark{1},
%Homer Simpson\IEEEauthorrefmark{2},
%James Kirk\IEEEauthorrefmark{3},
%Montgomery Scott\IEEEauthorrefmark{3} and
%Eldon Tyrell\IEEEauthorrefmark{4}}
%\IEEEauthorblockA{\IEEEauthorrefmark{1}School of Electrical and Computer Engineering\\
%Georgia Institute of Technology,
%Atlanta, Georgia 30332--0250\\ Email: see http://www.michaelshell.org/contact.html}
%\IEEEauthorblockA{\IEEEauthorrefmark{2}Twentieth Century Fox, Springfield, USA\\
%Email: homer@thesimpsons.com}
%\IEEEauthorblockA{\IEEEauthorrefmark{3}Starfleet Academy, San Francisco, California 96678-2391\\
%Telephone: (800) 555--1212, Fax: (888) 555--1212}
%\IEEEauthorblockA{\IEEEauthorrefmark{4}Tyrell Inc., 123 Replicant Street, Los Angeles, California 90210--4321}}




% use for special paper notices
%\IEEEspecialpapernotice{(Invited Paper)}




% make the title area
\maketitle

% As a general rule, do not put math, special symbols or citations
% in the abstract
\begin{abstract}
Nowadays, the ethernet is developing much faster than memory and CPU technologies, protocol processing has become the bottleneck of TCP performance on end systems. Modern NICs usually support offload techniques such as checksum offload and TCP Segmentation Offload(TSO), allowing the end system to offload some processing work onto the NIC hardware. In this paper, we present an implementation of Large Receive Offload(LRO) on a multi-core NPU platform. Experiment results demonstrate that we achieved \% performance gain compared to the Linux kernel implementation of LRO.
\end{abstract}

% no keywords




% For peer review papers, you can put extra information on the cover
% page as needed:
% \ifCLASSOPTIONpeerreview
% \begin{center} \bfseries EDICS Category: 3-BBND \end{center}
% \fi
%
% For peerreview papers, this IEEEtran command inserts a page break and
% creates the second title. It will be ignored for other modes.
\IEEEpeerreviewmaketitle



\section{Introduction}
% no \IEEEPARstart
%This demo file is intended to serve as a ``starter file''
%for IEEE conference papers produced under \LaTeX\ using
%IEEEtran.cls version 1.8b and later.
% You must have at least 2 lines in the paragraph with the drop letter
% (should never be an issue)
%I wish you the best of success.

%\hfill mds

%\hfill August 26, 2015
TCP is one of the most important network protocols and is extremely widely used, improving TCP performance can reduce server's cluster scale and computation power consumption, brings both commercial and environmental benefits. In the last few years, ethernet bandwidth has increased from 1Gbps to 100Gbps, while memory bandwidth from to , and CPU processing power only from Hz to Hz, the performance gap makes memory access and protocol processing become the bottleneck of TCP, instead of link capacity. The constantly increasing network bandwidth has caused a severe burden for CPU, optimizing TCP processing mechanism can mitigate the situation and improve end system TCP performance.

Traditional TCP acceleration techniques such as checksum optimization, zero-copy and interrupt coalescing are focused on the host side, protocol processing is still done by host CPU. The idea of offloading some TCP processing workload from end host to NIC hardware naturally came along. An extreme form of this idea is TOE, it can offload the entire TCP protocol processing workload and dramatically improve the end system TCP performance, but its implementation is very complex and it can cause security and compatibility issues. TSO optimizes TCP data sending path, it offloads user data segmentation and checksum calculation, the technique has become rather mature because of its simplicity, the Linux operating system now offers programming interfaces and developers can implement TSO on their NICs with little extra coding. LRO aggregates consecutive TCP data packets into large ones, the reformed packets are then forwarded to kernel network stack for further processing, LRO improves TCP performance by reducing the number of packet headers processed by CPU, but it works in the NIC driver layer and the packet aggregation job is still done by host CPU.

A multi-core NPU usually has excellent packet processing performance for the following reasons:
\begin{enumerate}
\item More than a dozen hardware based, low-switching-overhead threads. The large number of hardware contexts enables software to more effectively leverage the inherent parallelism exhibited by packet processing applications. When one hardware thread is waiting for memory access result, other threads could switch in and make memory access requests without much overhead, this pipelined mechanism hinds DRAM latency and increases the effective bandwidth.
\item Favorable I/O features. A multi-core NPU can import packets from interface to memory with high throughput, moreover, its dispatching mechanism can distribute packets to different threads or cores according to application configurations. The dispatching component could pipeline with corresponding processing threads and has very high flexibility.
\item Well designed message passing mechanism among different threads. A multi-core NPU often employs cross-bar structure or SRAM as its message transfer medium, which makes thread synchronization efficient and elegant.
\end{enumerate}

Different TCP flows are weakly correlated and can be processed concurrently, this naturally leads to the idea of employing a multi-core NPU's excellent packet processing capability to accelerate TCP processing on an end system. In this paper, we propose to use multi-core NPU as NIC and implement LRO on it, our implementation reduces the number of packets processed by network stack and the number of interrupts generated by NIC, eventually improves TCP performance on an end system. The experiment results demonstrate the effectiveness of our proposal. Further more, our implementation only involves the NIC hardware and driver layer, user applications and kernel network stack see no difference between the multi-core NPU and a normal NIC, our implementation does not suffer TOE's compatibility and security problems.

The rest of the paper is organized as follows:
\section{Related Work}

\section{System Architecture}
In our proposed scheme, we use multi-core NPU as NIC and implement LRO on it to accelerate TCP processing on the data receiving path. Packets reordering, packets aggregation and checksum verification(calculation) are done by multi-core NPU, NIC driver program is responsible for forwarding the reconstructed big packet to the kernel network stack. Our system resides in the NIC hardware and driver layer, as show in , framework of our system is depicted in .

As a NIC, the multi-core NPU must be able to send and receive packets, its numerous threads are divided into three categories: packet sending threads, packet receiving threads and a packet receiving timeout checking thread(we will explain the reason of its existence in the next section).

The sending process of a packet is quite similar to a regular NIC, the multi-core NPU waits for packets to be sent, calculate checksum for TCP/IP packets, then transmits them via the MAC component.

The receiving process of an ethernet packet, however, is much more complex. The multi-core NPU needs to check if a packet is suitable for LRO, handle out-of-order packets, check packet receiving timeout for a TCP flow, and reconstruct accumulated TCP data packets(we will describe these functionalities in more detail later). The packet receiving process sequence is show in : it starts by checking timeout messages, these messages are sent by a particular timeout checking thread, a timeout message indicates which TCP flow has paused(stopped) receiving new packets, and causes the multi-core NPU to aggregate buffered data packets of this flow. Then, the multi-core NPU checks for a newly arrived packet and runs a series of tests to see if the packet fits LRO requirements. If the packet passes those tests, it is buffered according to which TCP flow it belongs to and reordered by its sequence number for further packets reconstruction; if it failed LRO tests, it will be forwarded directly to the NIC driver program(e.g. a UDP packet) or simply discarded(e.g. incorrect checksum).

Since the multi-core NPU has done nearly all the dirty work, the NIC driver program's job is rather simple and straight-forward, very like other NIC driver programs except that: the driver program needs to pre-allocate consecutive pages for storing aggregated packet's data, and construct a correct skbuff data structure for it.
\section{System Functionalities}
The last section briefly introduced architecture of our system, now we will describe system functionalities in more detail.
\subsection{TCP Connection Management}
Our system does not offload all functions of TCP, so the multi-core NPU does not have to maintain all the information of a TCP connection like a TOE or the kernel protocol stack does, it only needs to maintain information which is necessary for packets reordering and reconstruction. Based on this concept, we employed two data structures for TCP connection management: ConnectionDescriptor and ConnectionTable.

ConnectionDescriptor is used to represent a certain TCP connection, it contains information including: the four-tuple(source IP address, destination IP address, source port number and destination port number), which can uniquely identify a TCP connection; the number and total data length of buffered packets; and the maximum ACK number after packets reordering.

Each packet receiving thread of the multi-core NPU needs to maintain multiple TCP connections, i.e., multiple ConnectionDescriptors, thus a fast searching mechanism is inevitable. ConnectionTable is introduced to fulfill this requirement, it is designed as a hash table, and used for searching the corresponding ConnectionDescriptor when a TCP packet arrives. Our system utilizes the multi-core NPU's packet dispatch mechanism, we extract the four-tuple information from TCP packets and distribute packets belonging to the same connection to the same packet receiving thread, when a packet receiving thread receives a TCP packet, it calculates hash value based on the packet's four-tuple, and looks for the corresponding ConnectionDescriptor via ConnectionTable. Each packet receiving thread maintains an independent ConnectionTable, so the search and update operations do not need to interact with other threads, synchronization overheads such as locking and unlocking are naturally avoided. ConnectionTable uses lists to resolve collision problems, its memory layout is shown in .

Our system monitors TCP's three-way handshake sequence, when the multi-core NPU receives a SYN packet, it considers a TCP connection is being established. The packet receiving thread gets a free ConnectionDescriptor, initializes it with SYN packet's four-tuple information and adds it to the ConnectionTable.

Our system also monitors FIN packets to see if a connection is being tared down. When a packet receiving thread receives a FIN packet, it searches for the corresponding ConnectionDescriptor, aggregates its buffered packets and generates an interrupt to make the driver program process this reconstructed packet. Then the ConnectionDescriptor is removed from ConnectionTable and marked as free for future use.

As with ConnectionTable, each packet receiving thread maintains its ConnectionDescriptors in an independant memory space to avoid thread synchronization operations. In addition, free ConnectionDescriptors are pre-allocated and appended to a queue, so a ConnectionDescriptor can be obtained and released through simple queue operations like enqueue and dequeue, instead of expensive dynamic memory allocate and free operations.
\subsection{Packets Filtering}
Each packet receiving thread of the multi-core NPU runs a series of tests to see if a packet is appropriate for reordering and aggregation operations, packets which failed these tests are discarded or forwarded to the driver program, packets which passed these tests will be buffered on their corresponding ConnectionDescriptor for further processing. The detailed filter sequence is show in , note that we do not require packets' TCP sequence number to be consecutive.
\subsection{Packets Reordering}
Each ConnectionDescriptor contains a list for buffering TCP data packets, multi-core NPU sorts packets by their TCP sequence number and stores them in increasing order. Pseudo code of our packets reordering algorithm is show in .
\floatname{algorithm}{Procedure}
\subsection{Packets Reconstruction}
\subsection{Timeout Check for Packet Receiving}
\section{Critical Improvement Techniques}
In this section, we will
\section{Implementation and Performance Evaluation}
% An example of a floating figure using the graphicx package.
% Note that \label must occur AFTER (or within) \caption.
% For figures, \caption should occur after the \includegraphics.
% Note that IEEEtran v1.7 and later has special internal code that
% is designed to preserve the operation of \label within \caption
% even when the captionsoff option is in effect. However, because
% of issues like this, it may be the safest practice to put all your
% \label just after \caption rather than within \caption{}.
%
% Reminder: the "draftcls" or "draftclsnofoot", not "draft", class
% option should be used if it is desired that the figures are to be
% displayed while in draft mode.
%
%\begin{figure}[!t]
%\centering
%\includegraphics[width=2.5in]{myfigure}
% where an .eps filename suffix will be assumed under latex,
% and a .pdf suffix will be assumed for pdflatex; or what has been declared
% via \DeclareGraphicsExtensions.
%\caption{Simulation results for the network.}
%\label{fig_sim}
%\end{figure}

% Note that the IEEE typically puts floats only at the top, even when this
% results in a large percentage of a column being occupied by floats.


% An example of a double column floating figure using two subfigures.
% (The subfig.sty package must be loaded for this to work.)
% The subfigure \label commands are set within each subfloat command,
% and the \label for the overall figure must come after \caption.
% \hfil is used as a separator to get equal spacing.
% Watch out that the combined width of all the subfigures on a
% line do not exceed the text width or a line break will occur.
%
%\begin{figure*}[!t]
%\centering
%\subfloat[Case I]{\includegraphics[width=2.5in]{box}%
%\label{fig_first_case}}
%\hfil
%\subfloat[Case II]{\includegraphics[width=2.5in]{box}%
%\label{fig_second_case}}
%\caption{Simulation results for the network.}
%\label{fig_sim}
%\end{figure*}
%
% Note that often IEEE papers with subfigures do not employ subfigure
% captions (using the optional argument to \subfloat[]), but instead will
% reference/describe all of them (a), (b), etc., within the main caption.
% Be aware that for subfig.sty to generate the (a), (b), etc., subfigure
% labels, the optional argument to \subfloat must be present. If a
% subcaption is not desired, just leave its contents blank,
% e.g., \subfloat[].


% An example of a floating table. Note that, for IEEE style tables, the
% \caption command should come BEFORE the table and, given that table
% captions serve much like titles, are usually capitalized except for words
% such as a, an, and, as, at, but, by, for, in, nor, of, on, or, the, to
% and up, which are usually not capitalized unless they are the first or
% last word of the caption. Table text will default to \footnotesize as
% the IEEE normally uses this smaller font for tables.
% The \label must come after \caption as always.
%
%\begin{table}[!t]
%% increase table row spacing, adjust to taste
%\renewcommand{\arraystretch}{1.3}
% if using array.sty, it might be a good idea to tweak the value of
% \extrarowheight as needed to properly center the text within the cells
%\caption{An Example of a Table}
%\label{table_example}
%\centering
%% Some packages, such as MDW tools, offer better commands for making tables
%% than the plain LaTeX2e tabular which is used here.
%\begin{tabular}{|c||c|}
%\hline
%One & Two\\
%\hline
%Three & Four\\
%\hline
%\end{tabular}
%\end{table}


% Note that the IEEE does not put floats in the very first column
% - or typically anywhere on the first page for that matter. Also,
% in-text middle ("here") positioning is typically not used, but it
% is allowed and encouraged for Computer Society conferences (but
% not Computer Society journals). Most IEEE journals/conferences use
% top floats exclusively.
% Note that, LaTeX2e, unlike IEEE journals/conferences, places
% footnotes above bottom floats. This can be corrected via the
% \fnbelowfloat command of the stfloats package.




\section{Conclusion}
The conclusion goes here.




% conference papers do not normally have an appendix


% use section* for acknowledgment
\section*{Acknowledgment}


The authors would like to thank...





% trigger a \newpage just before the given reference
% number - used to balance the columns on the last page
% adjust value as needed - may need to be readjusted if
% the document is modified later
%\IEEEtriggeratref{8}
% The "triggered" command can be changed if desired:
%\IEEEtriggercmd{\enlargethispage{-5in}}

% references section

% can use a bibliography generated by BibTeX as a .bbl file
% BibTeX documentation can be easily obtained at:
% http://mirror.ctan.org/biblio/bibtex/contrib/doc/
% The IEEEtran BibTeX style support page is at:
% http://www.michaelshell.org/tex/ieeetran/bibtex/
%\bibliographystyle{IEEEtran}
% argument is your BibTeX string definitions and bibliography database(s)
%\bibliography{IEEEabrv,../bib/paper}
%
% <OR> manually copy in the resultant .bbl file
% set second argument of \begin to the number of references
% (used to reserve space for the reference number labels box)
\begin{thebibliography}{1}

\bibitem{IEEEhowto:kopka}
H.~Kopka and P.~W. Daly, \emph{A Guide to \LaTeX}, 3rd~ed.\hskip 1em plus
  0.5em minus 0.4em\relax Harlow, England: Addison-Wesley, 1999.

\end{thebibliography}




% that's all folks
\end{document}


